\documentclass[answers]{exam}

\usepackage{amsmath}
\usepackage{amssymb}
\usepackage{geometry}
\usepackage{venndiagram}
\usepackage{graphics}
\usepackage{cancel}
\usepackage{hyperref}

% Header and footer.
\pagestyle{headandfoot}
\runningheadrule
\runningfootrule
\runningheader{CS 351 Artificial Intelligence}{Assignment 4 - Strength and Storms}{Fall 2022}
\runningfooter{}{Page \thepage\ of \numpages}{}
\firstpageheader{CS 351 Artificial Intelligence}{Assignment 4 - Strength and Storms}{Fall 2022}

\title{Assignment 4 - Strength and Storms\\ CS 351 Artificial Intelligence\\ Habib University -- Fall 2022}
\author{Ali Asghar Yousuf \\ ay06993}  % replace with your ID, e.g. oy02945
\date{\today}

\begin{document}
\maketitle

\begin{questions}
    \question You will find online a detailed study report on \href{https://ai100.stanford.edu/sites/g/files/sbiybj18871/files/media/file/AI100Report_MT_10.pdf}{The One Hundred Year Study on Artificial Intelligence} published by Stanford University.
    This study was conducted by a standing committee that involved renowned scientists in the field of AI who have provided their insiders' perspective on twelve different study questions. You are required to go over this report and express your opinion on the following questions:
    \begin{parts}
        \part Which event, in the history of AI, has been the most ground-breaking?
        \begin{solution}
            \textbf{Language Processing} is arguably the most ground-breaking area of Artificial Intelligence. Although learning meanings and facts may seem simple enough but in the context of Language Processing they are just a small part of the puzzel, understanding grammar and the context in which the words are used
            are the bigger chunk of the problem as texts from different times have different context for the same set of words which makes it even more difficult to achieve high level of proficiency in this field. However, the models we have today are far from being perfect, yet it is amazing how far we have come.
        \end{solution}

        \part What are the most promising opportunities for AI?
        \begin{solution}
            In my opinion \textbf{Decision Making} is definitely one of the most (if not the most) promising opportunities for AI. Although it is a quite controversial use of AI as people often consider decision making as subjective, but if broken into smaller problems it can be as objective as any other CS problem
            (I understand many people may disagree with this). And with the strives we are making in several fields including medicine, cosomology, etc. often in order to make a decision, there needs to be huge amounts of data to be analyzed, so with AI decision making I believe can be a huge step forward.
        \end{solution}
        \part What are the most pressing dangers of AI?
        \begin{solution}
            As AI is becoming more and more common in our daily world, it is inevitable to talk about the dangers of AI. As AI algorithms are trained on existing data and information from our soceity and since our society is not perfect, it is possible in systems that are designed to overcome biases from society may well
            worsen the pre existing situation.

            Similarly using \textbf{Statistical Perspective on Justice} may generate
            results that are subject to even more biases as discussed earlier.
        \end{solution}
        \part How much have we progressed in understanding the key mysteries of human
        Intelligence?
        \begin{solution}
            Both cognitive science, which is the study of human intelligence, and AI have significant effects on how we perceive human intellect. The fundamental questions of human intelligence that have baffled researchers for more than a decade include not only how we can solve complex problems
            and make rational decisions quickly, but also how we can handle emotionally charged situations in interpersonal interactions, use attitudes, emotions, and other bodily cues to guide our decisions, and understand the intentions of others. More than cognitive psychology or psychometrics, computer modelling,
            collective intelligence, and cognitive neuroscience have advanced our understanding of human intelligence over the past five years. The investigation of collective intelligence looks at how intelligence develops in group settings as opposed to in solitude. The field of cognitive neuroscience studies how the physical
            structure of the brain affects how psychological and social processes are carried out. Models of language processing, image recognition, and other cognitive processes are used in computational modelling, which is influenced by machine learning. The study of intelligence stresses how people may change, adapt,
            and prosper rather than solely concentrating on the operation of a potent information-processing system. There is still much to discover regarding the nature of consciousness, the functioning of the human mind, and the manner in which information from many senses, modalities, and sources is combined.

        \end{solution}
        \part How do you see AI transforming the society in positive or negative ways?
        Discuss.
        \begin{solution}
            A lot of progress has been achieved in recent years on several difficult problems that have been the focus of AI research, such real-time voice recognition and answering questions based on textbook reading. Technologies based on machine learning have entered the real world in a variety of ways. Neural network
            language models are utilised by a variety of software applications, including chatbots, machine translation, text categorization, speech recognition, and writing aid. The use of image processing technology, which is now widely accessible, to recognise faces and create lifelike images of people is raising controversy
            all over the world. Although fully autonomous driving has not yet advanced at the projected rate, autonomous cars have started to function in some areas. Already, AI is capable of identifying a variety of eye and skin conditions, identifying cancers, and supporting clinical diagnostic evaluations. Beyond detecting
            fraud and enhancing cybersecurity, financial institutions are using AI to automate legal and compliance processes as well as detect money laundering. Systems of recommendations have a significant effect on how individuals use products, services, and information, but they also pose serious ethical issues.

            The capability to identify, track, and monitor people in real-time is also a
            privacy worry. There is a major worry that AI would cause people to lose their
            employment, yet AI assisting in repetitive and time consuming tasks can save
            alot of time.
        \end{solution}
    \end{parts}

    \question
    Do some research to familiarize yourself with prominent scientists in the history of AI.
    You are required to describe the role of any four of the following scientists and their key contribution in
    pioneering the field.
    \begin{parts}
        \part Alan Turing
        \begin{solution}
            Alan Turing is hailed as the father of modern computer science. He invented the computer while attempting to solve a fiendish puzzle known as the Entscheidungs problem. He imagined a machine with an infinitely long tape with symbols on it that feeds instruction to the machine. This machine is the mathematical model
            of the modern computers in use today.

            Turing is best known for cracking the German Engima Code during World War II
            and for presenting a procedure known as Turing Test which laid the basis for
            AI.
        \end{solution}
        \part John McCarthy
        \begin{solution}
            John McCarthy was part of the team who coined the term ``Artificial Intelligence'' in a proposal at the famous Dartmouth conference in 1956. He developed the programming language Lisp which became the programming language of choice for AI applications after its publication in 1960. He significantly influenced the
            design of the programming language ALGOL, he proposed the use of recursion and conditional expressions, which became part of ALGOL. He invented garbage-collection; a method of automatic memory management to solve problems in Lisp.
        \end{solution}
        \part Marvin Minsky
        \begin{solution}
            The confocal microscope, a forerunner to the common confocal laser scanning microscope used today, and the first head-mounted graphical display were both created by Minsky in 1957 and 1963, respectively. The original Logo "turtle" was created by him and Seymour Papert. In 1951, he built the first computer for
            learning with a randomly wired neural network, known as SNARC. He developed a 7-state, 4-symbol computer in 1962 while working on small universal Turing machines. Minsky co authored the book Perceptrons which became the foundational work in the analysis of artificial neural networks.

            Minsky also worked on the famous movie 2001: A space Odyssey.
        \end{solution}
        \part Raymond Kurzwell
        \begin{solution}
            He works on areas including speech recognition, text-to-speech synthesis, optical character recognition (OCR), and electronic keyboard instruments. He has authored works on futuristic topics including health, artificial intelligence (AI), transhumanism, and the technological singularity.
            In 2010, he wrote and co-produced a movie directed by Anthony Waller The Singularity Is Near: A True Story About the Future, which was based in part on Kurzweil's 2005 book. Anthony Waller was the director. The movie is a hybrid of fiction and non-fiction, mixing interviews with 20 influential thinkers
            (including Marvin Minsky) with a storyline that exemplifies some of his main theories and features a computer avatar (Ramona) who rescues the Earth from self-replicating tiny robots. A feature-length independent documentary called Transcendent Man was also produced on Kurzweil, his life, and his views in addition
            to his film.
        \end{solution}
        \part Warren McCulloch
        \begin{solution}
            Together with Walter Pitts, McCulloch developed computational models based on threshold logic, which divided the investigation into two independent lines of research, one focusing on neural network applications to artificial intelligence and the other on biological processes in the brain.
            In their 1943 publication, McCulloch and Pitts made an effort to show that a Turing machine programme could be executed in a finite network of formal neurons, supporting their claim that the neuron served as the brain's fundamental logical unit. McCulloch began working at the Research Laboratory of Electronics at
            MIT in 1952 and focused mostly on modelling neural networks there. His team studied the frog's visual system in light of McCulloch's 1947 research and found that rather than only conveying a picture, the eye really gives the brain information that has already been organised and analysed to some extent.
            
        \end{solution}
        \part Judea Pearl
        \begin{solution}
            Judea Pearl is best known for supporting the advancement of Bayesian networks and the probabilistic approach to artificial intelligence, and a pioneer in the empirical sciences in the mathematization of causal modelling. He is also acknowledged for creating a theory of causal and counterfactual reasoning based on
            structural models.
        \end{solution}
    \end{parts}

\end{questions}

\end{document}

%%% Local Variables:
%%% mode: latex
%%% TeX-master: t
%%% End: